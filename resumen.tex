En las últimas décadas, nuestra capacidad de recolectar y almacenar grandes cantidades de datos observados ha aumentado enormemente. Esto hace que, una vez que se va a proceder a analizar y comprender estos datos, la cantidad de información de la que se dispone pueda ser abrumadora. En algunos casos, puede ser interesante realizar una reducción de la dimensión, esto es, descartar aquella información que es menos relevante y quedarnos con la verdaderamente significativa. La manera en la que realizamos esta selección es uno de los objetivos de la Estadística Multivariante.

La Estadística Multivariante es la rama de la Estadística que se ocupa de los datos multivariados, aquellos con dimensión mayor que 1. Una muestra de observaciones multivariadas se recoge siempre en una matriz. La descomposición en valores singulares (SVD por sus siglas en inglés) es una técnica del Álgebra Matricial que, en este contexto, permite resumir y comprimir la información contenida en la matriz de datos.

Una ventaja inmediata de aplicar la descomposición en valores singulares es facilitar el análisis de la información muestral, haciendo más sencilla la comprensión de los datos que se están manejando. El objetivo de este Trabajo Fin de Grado es estudiar cómo se utiliza concretamente la descomposición en valores singulares en diferentes técnicas multivariadas: componentes principales, correlaciones canónicas y aproximación y compleción de matrices. 

La base teórica y los procedimientos estudiados y demostrados se prueban después sobre un
conjuntos de datos reales con librerías del programa R. La aplicación satisfactoria de las herramientas estudiadas pretende demostrar la potencia y utilidad de la descomposición en valores singulares en datos actuales de diversas fuentes y temáticas.

\palabrasclave{Descomposición en valores singulares, Componentes principales, Correlaciones canónicas, Aproximación y compleción de matrices}